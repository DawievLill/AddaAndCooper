\section{Chapter 10: future developments}

\subsection{Optimal inventory policy}

The code \texttt{optimalInventory.py} solves the optimal inventory problem as described in Section 10.3.1 and Exercise 10.1 of \citet{adda2003dynamic}. The value function is given by (10.7):
%
\begin{equation}
	V(s, I) = \max_{y} \left\{ r(s) - c(y) + \beta EV(s', I') \right\}
\end{equation}
%

where \(r(\cdot)\) is revenue, a function of sales, \(s\), and \(c(\cdot)\) is the cost of producing \(y\) units of output, \(y\). \(I\) is the inventory held by the firm. The inventory evolves according to the following transition equation:
%
\begin{equation}
	I' = R(I + y - s)
\end{equation}
%
The functional forms used are:
%
\begin{align}
 r(s) &=  ps \\
 c(y) &= \eta y^{\gamma}
\end{align}
%
where \(p\) is the price and \(\eta\) is a scaling parameter.

The code \texttt{optimalInventoryShocks.py} solves the optimal inventory problem as described in Section 10.3.1 and Exercise 10.2 of \citet{adda2003dynamic}. The value function is given by (10.10):
%
\begin{equation}
	V(s, I, A) = \max_{y} \left\{ r(s) - c(y, A) + \beta EV(s', I', A') \right\}
\end{equation}
%
where \(A\) is a cost shock modelled as an AR(1) process and all other variables are as above. The functional form of the cost function is:
%
\begin{equation}
 c(y, A) = \eta A y^{\gamma}
\end{equation}
%
